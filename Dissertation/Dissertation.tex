%You can delete all the comments after you have finished your document
%this sets up the defaults for the documents, 12pt font and A4 size. The article type sets this up as such as opposed to letter or memo.

%for the finer points LaTeX see https://en.wikibooks.org/wiki/LaTeX or http://tex.stackexchange.com/

\documentclass[12pt,a4paper]{article}
\usepackage{titlesec} %these are how we import packages, one helps set up footers and title layout
\usepackage{fancyhdr}
\usepackage{titlesec}
\newcommand{\sectionbreak}{\clearpage}
\usepackage{apacite}
% !TEX TS-program = pdflatex
% !TEX encoding = UTF-8 Unicode
\usepackage[utf8]{inputenc} % set input encoding (not needed with XeLaTeX)
\usepackage{graphicx} % support the \includegraphics command and options

% \usepackage[parfill]{parskip} % Activate to begin paragraphs with an empty line rather than an indent

%%% PACKAGES
\usepackage{booktabs} % for much better looking tables
\usepackage{array} % for better arrays (eg matrices) in maths
\usepackage{paralist} % very flexible & customisable lists (eg. enumerate/itemize, etc.)
\usepackage{verbatim} % adds environment for commenting out blocks of text & for better verbatim
\usepackage{subfig} % make it possible to include more than one captioned figure/table in a single float
\usepackage[toc,page]{appendix}
% These packages are all incorporated in the memoir class to one degree or another...

%header and footer settings
\pagestyle{fancyplain}
\fancyhf{}
\renewcommand{\headrulewidth}{0.5pt}
\renewcommand{\footrulewidth}{0.5pt}
\setlength{\headheight}{15pt}
\fancyhead[L]{Matthew Lenathen - 40506678}
\fancyhead[R]{ SOC10101 Honours Project}
\fancyfoot[L]{}
\fancyfoot[C]{\thepage}

%this starts the document
\begin{document}

%you can import other documents into your main one, these layout the Title and Declarations on its own page.
%you might need to change these to \ if your on Microsoft Windows.
\input{./Dissertation-Title.tex}
\input{./Dissertation-Dec.tex}
\pagebreak
\input{./Dissertation-DP.tex}
\pagebreak

%LaTeX let you define the abstract separately so it wont get sucked into the main document.
\begin{abstract}
Abstract here
% fill the abstract in here
\end{abstract}
\pagebreak

\tableofcontents % is generated for you
\newpage

\listoftables
%generated in same way as figures
\newpage

\listoffigures
%you may have captions such as equations, listings etc they should all appear as required
%these are done for you as long as you use \begin{figure}[placement settings] .. bla bla ... \end{figure}
\newpage

\section*{Acknowledgements}
Insert acknowledgements here
\subsection*{}
	I would like to thank my cat, dog and family.
\newpage

\section{Introduction}
\newpage
\section{Literature Review}
\subsection{Introduction to Cloth Simulation}
Cloth is a major aspect of realistic simulations. Present in many different industries such as: film, games development and virtual reality, cloth simulation is a challenge that combines maths, physics and programming. \\

Cloth simulation is a subsection of Physics-based animation. PBA uses physics and code to create a plausible looking display of realistic effects, e.g. fluid simulations, soft-body dynamics etc. Cloth is defined as a flexible plane of some material or fabric that can move in unique ways, such as bending or stretching. This makes it a tricky and unique problem to simulate accurately. \\

The end goal of a cloth simulation can vary depending on the intended use. In computer graphics, physical accuracy is not the highest priority, the end product is mostly focused on how it looks. This is in contrast to cloth simulations for real-world engineering problems, their focus being primarily on physical accuracy, in order to help them predict real-world scenarios. Real time usage of cloth is also very important in the games industry, and will be discussed later. \\

As it is featured in such a wide range of fields, the literature relating to cloth is plentiful. In this section, many papers, books and articles will be discussed and compared in order to give context to the focus of this project, Extended position based dynamics (XPBD).

\subsection{Brief History of Cloth Simulation}
Cloth simulation is a very active field in research, and has been for a handful of decades. The first graphic models used to mimicking cloth was developed in 1986 by Jerry Weil. Before this, cloth objects were created by mapping textures onto rigid surfaces. His method represents cloth as a 2d grid of 3d coordinates, and focused on simulating cloth held at constraint points, like a draped fabric. \cite{weil_synthesis_1986}. This method didn't allow for cloth movement but it piqued the interest of the graphics community regarding cloth simulation. \\

An important development was made in 1987 by Terzopoulos et al. Their method was a general solution for representing elastic deformable objects like rubber or paper. This method is physically-based, meaning the object is active and can respond to external forces such as gravity or wind. Their paper presented various equations on how objects such as cloth or paper should deform when presented with forces, it also describes how they integrated these equations through time, which is a key topic in simulation. This research laid the groundwork for many more physically based models, and more cloth oriented methods. \cite{terzopoulos1987elastically} \\

As time went on, more ways to simulate cloth were developed, such as particle based systems by Breen et al. in 1992. In particle-based systems, the simulating focused on a series of connected particles, and how they react with each other. The way these particles interacted formed the basis for the cloth dynamics. \cite{breen1992physically} In their paper they decided to move away from continuum based methods and switch towards a particle based model. In 1995, Provot developed a similar particle system but used point masses connected with springs to represent a cloth mesh. \cite{provot1995deformation} \\

In 1998, Baraff and Witkin developed a method that used implicit integration combined with the cloth being represented as a mesh of triangles. This new proposed system allowed for bigger time steps to be taken and faster simulation times. This system is still used today as the foundation for new methods. \cite{Baraff1998largesteps} \\

Later, in 2007, Müller et al. developed and published a paper on a state of the art method for dynamic simulations called Position Based Dynamics. This technique, along with a newer extended version, is the main focus of this project and will be discussed in a later chapter. \cite{muller2007position} and \cite{macklin2016xpbd}

\subsection{Offline Cloth Simulation}
Cloth simulation can generally by split into two categories: Offline and Real Time. Offline simulations are calculated, tweaked and edited before being rendered to the screen. Commonly used in film and animation, visual fidelity is of utmost importance. Offline calculations can also help in a real time scenario by precomputing effects. \\

\subsubsection{Fine Cloth Interactions}
As offline simulations are not limited by the time between each frame, they can afford to focus on realism. They can do this by scaling the number of triangles up and allowing for finer interactions like folding and wrinkling, these interactions being crucial to a convincing cloth simulation. A great example of such work is presented in a paper by Selle et al., they detail their method of creating a high resolution simulation with up to two million triangles, that can accurately capture the intricate details of cloth. \cite{4522545}\\

At the time of publication, cloth simulations were adding physically-based interactions like wrinkles through a separate modelling system. This can be seen in the work by Cutler et al., in which they created a system that allowed the artist to add in wrinkles on top of an existing character's clothing \cite{10.1145/1073368.1073384}. They achieved this by splitting it into stages: wrinkle creation and wrinkle evaluation. In the wrinkle creation phase, the artists use a tool to create winkle patterns, which get turned into a set of forces on the surface. Next, in the wrinkle evaluation phase, the system generates these deformations based on character poses, and the artists can either choose to keep these generated wrinkles, or alter the pattern until satisfactory results are achieved.\\
In their results, they were able to use their system to create clothing wrinkles on hundreds of characters for a feature-length animated film. They noted that in production, close to zero hand-tweaking was needed for the wrinkles, proving its reliability.\\
When discussing their work, they did note that the visual results from a dynamic clothing simulation is much preferred to what they achieved with their kinematic wrinkle system. This is where the work of Selle et al., aims to improve upon. \\



----------------
\subsubsection{Precomputing Effects}
In computer graphics, it is often said that precomputing everything is simply not viable due to how many interactions can occur, but in a paper by Kim et al., they discuss precomputing secondary cloth effects. \cite{kim2013near}. In this case, secondary cloth effects refer to things such as folds and wrinkles, these are expensive simulate in real-time. In this paper, they propose a method to use many thousand CPU-hours to calculate these advanced effects in advance, based on a character motion graph. They achieve this by representing a character with two motion graphs, the primary graph simply being the movement of the character, and the secondary being the cloth dynamics. \\

To evaluate their method, they acquired 12 unique motion clips for the primary graph, and used ARCSim cloth simulator for the cloth motion. They found that in their massive precomputation, they were able to generate good looking cloth animations for any random paths through the primary graph. Initially their secondary graph was just one to one mapping of character poses to cloth, but this proved insufficient for realistic motion, so the secondary graph was made much more complex. 
Overall, the inclusion of precomputing cloth motion, helped simulate incredibly detailed secondary features of cloth, such as wrinkles and folds. This research also links into game development in their discussion section, they propose that viewing a database of precomputed cloth motion could be a great benefit when creating games, as certain scenes could be viewed and edited, also bad looking cloth could be fixed by hand to help realism within the game.


\subsection{Real Time Cloth Simulation}
Real time cloth simulations are created by computing the cloth dynamics at runtime, allowing interaction with the user through input and changes from the environment. As the time between frames on real time applications is small, the dynamics need to be calculated and applied efficiently, usually on the GPU. The most common use case for real time cloth simulations are in video games, and as these games also have lots of other calculations going on in the background, means optimisation is even more important. This section will discuss the literature on methods to create these simulations and how they compare with each other. \\
 
\subsubsection{Underlying Physics}
In order to discuss the advancements and methods within the current literature, it is important to delve into the underlying physics that allow cloth simulations to function. This section will go over important principles that are crucial to creating such simulations.
\\

Often, cloth is represented by particles. These particles have position p, and velocity v, and can be arranged in a grid structure to model a piece of cloth. In order to represent the actual geometry of these particles, they can be arranged into triangles that make up a bigger mesh. When these triangles are small enough, that is when the simulated cloth can look realistic.
\\

For simulations such as mass-spring systems, forces are the primary mechanism for making the particles move. In this case, spring forces between two particles are calculated using Hooke's law, and they say that if the spring is stretched or compressed more than the springs rest length, forces are applied to remedy that. Other common forces are gravity and wind and these will be applied to near all simulations regardless of if it is a force-based method or not.
\\

As the main focus of this project is position-based dynamics, constraints must be discussed too. Constraints are pivotal to position-based dynamics because instead of relying on forces, PBD directly adjusts positions to satisfy constraints. Constraints can vary depending on the situation but usually there are some staples such as distance, bending and collision constraints. Each of these will ensure that the particles within the simulation act as realistic as possible. 


\subsubsection{Particle-Based Systems}
\subsubsection{Mass-Spring Systems}
\subsubsection{Continuum}
\subsubsection{Alternative Methods for Cloth Simulation}
\subsubsection{Position Based Dynamics}
\subsubsection{Extended Position Based Dynamics}




\subsection{Future Trends}
\subsection{Conclusion}

\newpage



\section{Rest of Diss}

\subsubsection{Overview Of Project Content and Milestones}

This is a sub sub section with a list of bullet points.
\begin{itemize}\itemsep0pt
	\item A working X, that will be used for this investigation.
	\item Investigation of current tools and their potential use during an investigation of X .
	\item Programming of X with related frameworks Y and Z.
	\item That is all.
\end{itemize}


\section{Chapter 2}
The following bibliographic information on Writing a literature review is contained in the \emph{bibliography.bib} file 
The template automatically starts new chapters on a new page.  The associated guidelines tell you what the available styles do and also how to structure a report.
There is a section break on this page that you should be careful NOT to delete otherwise the references and appendices will be numbered continuously with the rest of the document.

% another example section
\section{Additional Information / Knowledge Required}
Experience with Linux and managing Virtual machines, networking.
So on and so forth...


\bibliographystyle{apacite}
\bibliography{bibliography}

%you can crate this on a extra tex document just like the title or any other part of the document.
\newpage
\begin{appendices}
\section{Project Overview}
%insert IPO

\begin{subappendices}
\subsection{Example sub appendices}
...
\end{subappendices}

\section{Second Formal Review Output}
Insert a copy of the project review form you were given at the end of the review by the second marker

\section{Diary Sheets (or other project management evidence)}
Insert diary sheets here together with any project management plan you have

\section{Appendix 4 and following}
insert content here and for each of the other appendices, the title may be just on a page by itself, the pages of the appendices are not numbered, unless an included document such as a user manual or design document is itself pager numbered.
\end{appendices}

\end{document}
